\documentclass[a4paper, 11pt, UTF8]{article}
%\documentclass[UTF8]{ctexbook}
%%%%%% 导入包 %%%%%%
\usepackage{CJKutf8}
\usepackage{graphicx}
\usepackage[unicode]{hyperref}
\usepackage{xcolor}
\usepackage{cite}
\usepackage{indentfirst}
\usepackage[strict]{changepage}

\usepackage{titlesec}   %设置页眉页脚的宏包
\usepackage{fancyhdr}
\usepackage{lastpage}
\usepackage{layout}
\usepackage{CTEX}

%%%% 下面的命令重定义页面边距,使其符合中文刊物习惯 %%%%
\addtolength{\topmargin}{-54pt}
\setlength{\oddsidemargin}{0.63cm}  % 3.17cm - 1 inch
\setlength{\evensidemargin}{\oddsidemargin}
\setlength{\textwidth}{14.66cm}
\setlength{\textheight}{24.00cm}    % 24.62

%\pagestyle{empty}                   %不设置页眉页脚
\footskip = 10pt
\pagestyle{fancy}                    % 设置页眉
\begin{CJK}{UTF8}{gkai}\end{CJK} % gbsn(宋体), gkai(楷体)
%\lhead{page \thepage\ of \pageref{LastPage}}
\lhead{\hspace{0.3cm} 标题}
%\rhead{\small\leftmark}
%\rhead{\thepage \hspace{0.3cm}}
\lhead{\hspace{0.1cm}}
\cfoot{\thepage}
%\rfoot{页脚左}%
%\lfoot{页脚右}
\renewcommand{\headrulewidth}{0.8pt}  %页眉线宽,设为0可以去页眉线
\setlength{\skip\footins}{0.5cm}    %脚注与正文的距离
%\renewcommand{\footnotesize}{}      %设置脚注字体大小
%\renewcommand{\footrulewidth}{0.8pt}  %脚注线的宽度
%%==============
%%双线页眉的设置
%\makeatletter %双线页眉
%\def\headrule{{\if@fancyplain\let\headrulewidth\plainheadrulewidth\fi%
%\hrule\@height 1.0pt \@width\headwidth\vskip1pt%上面线为1pt粗
%\hrule\@height 0.5pt\@width\headwidth  %下面0.5pt粗
%\vskip-2\headrulewidth\vskip-1pt}      %两条线的距离1pt
%\vspace{6mm}}     %双线与下面正文之间的垂直间距
%\makeatother
%===============
%\pagestyle{fancy}
%\fancyhead{} %clear all fields
%\fancyhead[CE]{ \应 \用 \数 \学 }
%\fancyhead[CO]{{ Roll920: 模板}}
%\fancyhead[RO]{\thepage} %奇数页眉的右边
%\fancyhead[LE]{\thepage} %偶数页眉的左边
%\fancyhead[RE]{\zihao{-5} 2005 c}
%\fancyfoot[C]{}
%%%%%% 设置字号 %%%%%%
\newcommand{\chuhao}{\fontsize{42pt}{\baselineskip}\selectfont}
\newcommand{\xiaochuhao}{\fontsize{36pt}{\baselineskip}\selectfont}
\newcommand{\yihao}{\fontsize{28pt}{\baselineskip}\selectfont}
\newcommand{\erhao}{\fontsize{21pt}{\baselineskip}\selectfont}
\newcommand{\xiaoerhao}{\fontsize{18pt}{\baselineskip}\selectfont}
\newcommand{\sanhao}{\fontsize{15.75pt}{\baselineskip}\selectfont}
\newcommand{\sihao}{\fontsize{14pt}{\baselineskip}\selectfont}
\newcommand{\xiaosihao}{\fontsize{12pt}{\baselineskip}\selectfont}
\newcommand{\wuhao}{\fontsize{10.5pt}{\baselineskip}\selectfont}
\newcommand{\xiaowuhao}{\fontsize{9pt}{\baselineskip}\selectfont}
\newcommand{\liuhao}{\fontsize{7.875pt}{\baselineskip}\selectfont}
\newcommand{\qihao}{\fontsize{5.25pt}{\baselineskip}\selectfont}
\newcommand{\HRule}{\rule{\linewidth}{0.5mm}}
\newcommand{\HRulegrossa}{\rule{\linewidth}{1.2mm}}

%%%% 设置 section 属性 %%%%
\makeatletter
\renewcommand\section{\@startsection{section}{1}{\z@}%
{-1.5ex \@plus -.5ex \@minus -.2ex}%
{.5ex \@plus .1ex}%
{\normalfont\sihao\CJKfamily{hei}}}
\makeatother

%%%% 设置 subsection 属性 %%%%
\makeatletter
\renewcommand\subsection{\@startsection{subsection}{1}{\z@}%
{-1.25ex \@plus -.5ex \@minus -.2ex}%
{.4ex \@plus .1ex}%
{\normalfont\xiaosihao\CJKfamily{hei}}}
\makeatother

%%%% 设置 subsubsection 属性 %%%%
\makeatletter
\renewcommand\subsubsection{\@startsection{subsubsection}{1}{\z@}%
{-1ex \@plus -.5ex \@minus -.2ex}%
{.3ex \@plus .1ex}%
{\normalfont\xiaosihao\CJKfamily{hei}}}
\makeatother

%%%% 段落首行缩进两个字 %%%%
\makeatletter
\let\@afterindentfalse\@afterindenttrue
\@afterindenttrue
\makeatother
\setlength{\parindent}{2em}  %中文缩进两个汉字位


%%%% 下面的命令设置行间距与段落间距 %%%%
\linespread{1.4}
% \setlength{\parskip}{1ex}
\setlength{\parskip}{0.5\baselineskip}

%%%% 正文开始 %%%%

\begin{document}
\begin{CJK}{UTF8}{gbsn} % gbsn(宋体) gkai(楷体)

%%%% 定理类环境的定义 %%%%
\newtheorem{example}{例}             % 整体编号
\newtheorem{algorithm}{算法}
\newtheorem{theorem}{定理}[section]  % 按 section 编号
\newtheorem{definition}{定义}
\newtheorem{axiom}{公理}
\newtheorem{property}{性质}
\newtheorem{proposition}{命题}
\newtheorem{lemma}{引理}
\newtheorem{corollary}{推论}
\newtheorem{remark}{注解}
\newtheorem{condition}{条件}
\newtheorem{conclusion}{结论}
\newtheorem{assumption}{假设}

%%%% 重定义 %%%%
\renewcommand{\contentsname}{目录}  % 将Contents改为目录
\renewcommand{\abstractname}{摘要}  % 将Abstract改为摘要
\renewcommand{\refname}{参考文献}   % 将References改为参考文献
\renewcommand{\indexname}{索引}
\renewcommand{\figurename}{图}
\renewcommand{\tablename}{表}
\renewcommand{\appendixname}{附录}
\renewcommand{\algorithm}{算法}


\begin{titlepage}
\begin{center}
% Upper part of the page
\textsc{\LARGE \LaTeX 中文模板}\\[0.5cm]
\textsc{\Large 模板说明文档}\\[3cm]
% Title
\HRule \\[0.4cm] { \huge \bfseries \LaTeX中文模板}\\ \HRule \\[3cm]
% Author and supervisor
\begin{minipage}{0.6\textwidth}
\begin{center}
副标题
\\[0.5cm]
作者: Roll920
\end{center}
\end{minipage}

\vfill
% Bottom of the page
{\large \today}
\end{center}
\end{titlepage}
\thispagestyle{empty}
\tableofcontents
\newpage
\clearpage

\section{章节1}
输入内容
这是一个CTEX的utf-8编码例子,{\kaishu 这里是楷体显示},{\songti 这里是宋体显示},{\heiti 这里是黑体显示},{\fangsong 这里是仿宋显示},{\lishu 这里是隶书显示},{\youyuan 这里是幼圆显示}。
\subsection{章节1.1}
输入内容~\cite{krizhevsky12nips}
\subsubsection{章节1.1.1}
输入内容

\section{章节2}

\section{章节3}

\section{总结}
\bibliographystyle{unsrt}
\bibliography{ref}

\clearpage
\end{CJK}
\end{document} 